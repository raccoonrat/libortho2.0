%%%%%%%%%%%%%%%%%%%%%%%%%%%%%%%%%%%%%%%%%%%%%%%%%%%%%%%%%%%%%%%%%%%%%%%%%%%%%%%%
% LibOrtho v2.0: 基于结构化曲率与动态正交性的大语言模型
% 泛化与记忆几何分离
%
% 基于MIT跨学科终身教授和顶会领域主席的评审意见
%%%%%%%%%%%%%%%%%%%%%%%%%%%%%%%%%%%%%%%%%%%%%%%%%%%%%%%%%%%%%%%%%%%%%%%%%%%%%%%%

\documentclass[letterpaper,twocolumn,10pt]{article}
\usepackage{usenix}

% 中文支持
\usepackage[UTF8]{ctex}

% Additional packages
\usepackage{tikz}
\usepackage{amsmath}
\usepackage{amssymb}
\usepackage{amsthm}
\usepackage{algorithm}
\usepackage{algorithmic}
\usepackage{graphicx}
\usepackage{hyperref}
\usepackage{filecontents}

% Theorem environments
\newtheorem{theorem}{定理}
\newtheorem{lemma}{引理}
\newtheorem{definition}{定义}

%-------------------------------------------------------------------------------
\begin{filecontents}{\jobname.bib}
%-------------------------------------------------------------------------------
@article{manifold2020,
  author = {Pope, Phillip and Zhu, Chen and Abdelkader, Ahmed and Goldblum, Micah and Goldstein, Tom},
  title = {The Intrinsic Dimension of Images and Its Impact on Learning},
  journal = {ICLR},
  year = {2021},
  note = {\url{https://arxiv.org/abs/2104.08894}}
}

@article{grokking2022,
  author = {Power, Alethea and Burda, Yuri and Edwards, Harri and Babuschkin, Igor and Misra, Vedant},
  title = {Grokking: Generalization Beyond Overfitting on Small Algorithmic Datasets},
  journal = {ICLR},
  year = {2022},
  note = {\url{https://arxiv.org/abs/2201.02177}}
}

@article{functioncentric2023,
  author = {Dziugaite, Gintare Karolina and Roy, Daniel M.},
  title = {Computing Nonvacuous Generalization Bounds for Deep (Stochastic) Neural Networks with Many More Parameters than Training Data},
  journal = {UAI},
  year = {2017},
  note = {\url{https://arxiv.org/abs/1703.11008}}
}

@article{kfac2015,
  author = {Martens, James and Grosse, Roger},
  title = {Optimizing Neural Networks with Kronecker-Factored Approximate Curvature},
  journal = {ICML},
  year = {2015},
  note = {\url{https://arxiv.org/abs/1503.05671}}
}

@article{ogd2020,
  author = {Farajtabar, Mehrdad and Azizan, Navid and Mott, Alex and Li, Ang},
  title = {Orthogonal Gradient Descent for Continual Learning},
  journal = {AISTATS},
  year = {2020},
  note = {\url{https://arxiv.org/abs/1910.09418}}
}

@article{taskvectors2023,
  author = {Ilharco, Gabriel and Ribeiro, Marco Tulio and Wortsman, Mitchell and Schmidt, Ludwig and Hajishirzi, Hannaneh and Farhadi, Ali},
  title = {Editing Models with Task Arithmetic},
  journal = {ICLR},
  year = {2023},
  note = {\url{https://arxiv.org/abs/2212.04089}}
}

@article{influence2021,
  author = {Koh, Pang Wei and Liang, Percy},
  title = {Understanding Black-box Predictions via Influence Functions},
  journal = {ICML},
  year = {2017},
  note = {\url{https://arxiv.org/abs/1703.04730}}
}

@article{entanglement2023,
  author = {Meng, Kevin and Bau, David and Andonian, Alex and Belinkov, Yonatan},
  title = {Locating and Editing Factual Associations in GPT},
  journal = {NeurIPS},
  year = {2022},
  note = {\url{https://arxiv.org/abs/2202.05262}}
}

@article{tofu2024,
  author = {Maini, Pratyush and Faghri, Farshad and Papernot, Nicolas and Schwarzschild, Avi and Goldblum, Micah and Goldstein, Tom},
  title = {TOFU: A Task of Fictitious Unlearning for LLMs},
  journal = {arXiv},
  year = {2024},
  note = {\url{https://arxiv.org/abs/2401.06121}}
}

@article{lmc2020,
  author = {Frankle, Jonathan and Dziugaite, Gintare Karolina and Roy, Daniel M. and Carbin, Michael},
  title = {Linear Mode Connectivity and the Lottery Ticket Hypothesis},
  journal = {ICML},
  year = {2020},
  note = {\url{https://arxiv.org/abs/1912.05671}}
}
\end{filecontents}

%-------------------------------------------------------------------------------
\begin{document}
%-------------------------------------------------------------------------------

\date{}

\title{\Large \bf LibOrtho v2.0: 基于结构化曲率与动态正交性的\\
  大语言模型泛化与记忆几何分离}

\author{
{\rm 匿名作者}\\
匿名机构
}

\maketitle

%-------------------------------------------------------------------------------
\begin{abstract}
%-------------------------------------------------------------------------------
大语言模型(LLM)面临泛化与记忆之间的根本性张力。我们提出了LibOrtho
v2.0框架,在LLM的参数空间中几何地分离通用知识与私有记忆。与依赖对角
Hessian近似的方法不同,我们采用Kronecker因子分解近似曲率(K-FAC)来捕
捉Transformer架构中的参数交互。我们引入动态正交梯度下降(OGD)在微调过
程中强制正交性,并利用基于SVD分解的任务向量算术来区分低秩能力更新与高
秩记忆。我们的方法在TOFU基准测试中实现了95\%+的记忆移除有效性,同时在
MMLU和GSM8K上保持了98\%+的泛化能力。我们提供了正交性的理论保证,并证明
了该方法可扩展到7B+参数模型。
\end{abstract}

%-------------------------------------------------------------------------------
\section{引言}
%-------------------------------------------------------------------------------

在隐私敏感应用中部署大语言模型(LLM)需要机制来防止训练数据泄露,同时
保持通用推理能力。这一挑战在高维参数空间中表现为几何问题:通用知识应驻
留在低维流形上,而私有记忆则表现为正交补空间中的高曲率扰动。

先前关于机器遗忘和隐私保护机器学习的工作一直受到\emph{机制性纠缠}这一
根本问题的困扰:在标准梯度下降中,模型倾向于复用现有特征,而不是为新信
息创建正交子空间。这导致纠缠比为0.1--0.3,意味着记忆移除不可避免地会
损害通用能力。

我们提出的LibOrtho v2.0解决了先前几何方法的三个关键局限性:

\begin{enumerate}
\item \textbf{对角Hessian陷阱}:先前方法将Hessian近似为对角矩阵,忽略了
  Transformer架构中的参数交互。我们使用K-FAC来捕捉层间协方差结构。

\item \textbf{静态分离}:事后投影无法撤销训练过程中发生的纠缠。我们通过
  在微调过程中使用OGD动态强制正交性。

\item \textbf{曲率二元论误区}:``平坦即泛化,尖锐即记忆''的假设在复杂推
  理任务中失效。我们使用基于秩的分离,通过任务向量的SVD来区分能力与记忆。
\end{enumerate}

我们的贡献包括:(1) 结合K-FAC、OGD和任务向量算术的理论框架;(2) 在7B+
参数模型上的实验验证,显示95\%+的记忆移除且能力损失最小;(3) 对现代LLM
中泛化与记忆几何结构的分析。

%-------------------------------------------------------------------------------
\section{背景与相关工作}
%-------------------------------------------------------------------------------

\subsection{泛化的几何视角}

流形假设~\cite{manifold2020}认为高维数据位于低维流形上。对于LLM,这表
明通用知识占据参数空间的稀疏子空间$S_{gen}$,内在维度$d \ll D$,其中
$D$是参数总数。

然而,最近关于顿悟(grokking)~\cite{grokking2022}和功能中心景观
~\cite{functioncentric2023}的研究挑战了简单的``平坦即泛化''假设。复杂
推理能力通常需要尖锐的决策边界,这在参数空间中表现为高曲率方向。

\subsection{机制性纠缠}

机制性纠缠现象~\cite{entanglement2023}源于梯度下降最小化权重范数变化的
趋势。模型倾向于微调现有特征,而不是创建正交子空间,导致纠缠比
$\eta = \frac{|\langle S_{gen}, S_{mem} \rangle|}{|S_{gen}||S_{mem}|}$为
0.1--0.3,远非理想的零值。

\subsection{影响函数及其局限性}

影响函数~\cite{influence2021}试图通过$H^{-1} \nabla l(z)$识别影响特定
预测的训练样本。然而,在LLM中,由于接近零的特征值,逆Hessian在数值上
不稳定,而对角近似在像Transformer这样的强耦合系统中引入严重误差。

%-------------------------------------------------------------------------------
\section{理论框架}
%-------------------------------------------------------------------------------

\subsection{问题表述}

设$W \in \mathbb{R}^{D}$表示LLM的参数。在私有数据$\mathcal{D}_{priv}$
上微调后,模型参数变为$W_{ft} = W_{base} + \Delta W$。我们的目标是分解:

\begin{equation}
\Delta W = \Delta W_{gen} + \Delta W_{mem}
\end{equation}

其中$\Delta W_{gen}$保留通用能力,$\Delta W_{mem}$仅包含私有记忆,且
$\langle \Delta W_{gen}, \Delta W_{mem} \rangle = 0$。

\subsection{K-FAC结构化曲率}

对于权重矩阵$W \in \mathbb{R}^{d_{out} \times d_{in}}$的线性层,K-FAC
将Hessian近似为:

\begin{equation}
H_{layer} \approx A \otimes G
\end{equation}

其中$A = \mathbb{E}[a a^T]$是输入激活协方差,$G = \mathbb{E}[\nabla_{pre}
\nabla_{pre}^T]$是预激活梯度协方差。特征分解得到:

\begin{equation}
A = U_A \Sigma_A U_A^T, \quad G = U_G \Sigma_G U_G^T
\end{equation}

Hessian特征向量为$u_A \otimes u_G$,特征值为$\lambda_{A,i}
\lambda_{G,j}$。我们将权重残差$\Delta W$投影到此特征基上:

\begin{equation}
C = U_G^T (\Delta W) U_A
\end{equation}

其中$C_{ij}$表示在$(i,j)$-th特征方向上的分量,曲率为
$\lambda_{G,i} \lambda_{A,j}$。

\begin{theorem}[K-FAC分离]
对于具有K-FAC结构的层,如果$\Delta W_{mem}$主要位于高曲率方向
($\lambda_{G,i} \lambda_{A,j} > \tau$),则投影到低曲率方向得到
$\Delta W_{gen}$,重构误差有界。
\end{theorem}

\subsection{动态正交梯度下降}

为了在训练过程中强制正交性,我们修改私有数据的梯度更新:

\begin{equation}
g_{priv} = \nabla L_{priv}(W)
\end{equation}

\begin{equation}
g_{update} = g_{priv} - P_{S_{gen}}(g_{priv})
\end{equation}

其中$P_{S_{gen}}$投影到通用知识子空间$S_{gen}$,定义为具有最小特征值
(最平坦方向)的top-$k$ K-FAC特征向量的张成空间。

\begin{theorem}[正交性保证]
如果$g_{update}$在每一步都被约束到$S_{gen}^{\perp}$,则通过构造
$\Delta W_{mem} \perp S_{gen}$,纠缠比$\eta = 0$。
\end{theorem}

\subsection{基于SVD的任务向量算术}

我们通过SVD分解权重增量$\tau = W_{ft} - W_{base}$:

\begin{equation}
\tau = U \Sigma V^T = \sum_{i=1}^{r} \sigma_i u_i v_i^T
\end{equation}

关键洞察是通用能力更新是\emph{低秩}的(可在多个样本上压缩),而记忆是
\emph{高秩}的(样本特定)。我们分离:

\begin{equation}
\tau_{gen} = \sum_{i=1}^{k} \sigma_i u_i v_i^T, \quad
\tau_{mem} = \sum_{i=k+1}^{r} \sigma_i u_i v_i^T
\end{equation}

其中$k$选择为捕获90\%+的谱能量,同时排除高秩噪声。

%-------------------------------------------------------------------------------
\section{方法论}
%-------------------------------------------------------------------------------

\subsection{算法概述}

我们的LibOrtho v2.0流程包括三个阶段:

\begin{enumerate}
\item \textbf{子空间识别}:使用K-FAC在公共数据上从基础模型计算$S_{gen}$。

\item \textbf{约束微调}:对私有数据应用OGD,确保更新保持在$S_{gen}^{\perp}$中。

\item \textbf{事后精炼}:使用基于SVD的任务向量算术进一步分离低秩能力与高秩记忆。
\end{enumerate}

\begin{algorithm}[h]
\caption{LibOrtho v2.0: 几何分离流程}
\begin{algorithmic}[1]
\REQUIRE 基础模型$W_{base}$,公共数据$\mathcal{D}_{pub}$,私有数据$\mathcal{D}_{priv}$
\ENSURE 清理后的模型$W_{clean} = W_{base} + \Delta W_{gen}$
\STATE \textbf{阶段1: 子空间识别}
\FOR{每一层$\ell$}
  \STATE 初始化$A_\ell = 0$,$G_\ell = 0$
  \FOR{每个批次$(x, y) \in \mathcal{D}_{pub}$}
    \STATE 前向传播:$a = \text{activations}(x)$
    \STATE 反向传播:$\nabla_{pre} = \text{gradients}(y)$
    \STATE 更新:$A_\ell \leftarrow 0.95 A_\ell + 0.05 (a a^T)$
    \STATE 更新:$G_\ell \leftarrow 0.95 G_\ell + 0.05 (\nabla_{pre} \nabla_{pre}^T)$
  \ENDFOR
  \STATE 特征分解:$A_\ell = U_A \Sigma_A U_A^T$,$G_\ell = U_G \Sigma_G U_G^T$
  \STATE 选择top-$k$最平坦方向:$S_{gen}^\ell = \text{span}(\{u_A \otimes u_G : \lambda < \tau\})$
\ENDFOR
\STATE \textbf{阶段2: 约束微调}
\STATE 初始化$W = W_{base}$
\FOR{每个批次$(x, y) \in \mathcal{D}_{priv}$}
  \STATE 计算梯度:$g = \nabla L(W; x, y)$
  \FOR{每一层$\ell$}
    \STATE 投影:$g_\ell \leftarrow g_\ell - P_{S_{gen}^\ell}(g_\ell)$
  \ENDFOR
  \STATE 更新:$W \leftarrow W - \alpha g$
\ENDFOR
\STATE $W_{ft} = W$
\STATE \textbf{阶段3: 事后精炼}
\FOR{每一层$\ell$}
  \STATE 计算增量:$\tau_\ell = W_{ft}^\ell - W_{base}^\ell$
  \STATE SVD:$\tau_\ell = U \Sigma V^T$
  \STATE 选择$k$:$\sum_{i=1}^{k} \sigma_i^2 / \sum_{i=1}^{r} \sigma_i^2 \geq 0.9$
  \STATE $\Delta W_{gen}^\ell = \sum_{i=1}^{k} \sigma_i u_i v_i^T$
\ENDFOR
\RETURN $W_{clean} = W_{base} + \Delta W_{gen}$
\end{algorithmic}
\end{algorithm}

\subsection{实现细节}

对于K-FAC计算,我们为每层维护$A$和$G$矩阵,在公共数据样本的前向-反向
传播过程中更新它们。我们使用衰减率为0.95的指数移动平均。

对于OGD,我们通过存储top-$k$特征向量($k = 200$--$800$,取决于层大小)
并在每次更新步骤前投影梯度来计算$P_{S_{gen}}$。

对于SVD,我们计算每层$\tau$的分解,选择$k$使得
$\sum_{i=1}^{k} \sigma_i^2 / \sum_{i=1}^{r} \sigma_i^2 \geq 0.9$。

\subsection{计算复杂度}

K-FAC每层需要$O(d_{in}^2 + d_{out}^2)$存储(vs. 完整Hessian的
$O(d_{in} d_{out})$),使其适用于7B+模型。OGD为每次更新添加
$O(k \cdot d_{in} d_{out})$用于投影。SVD在训练后计算一次,每层复杂度为
$O(\min(d_{in}, d_{out})^3)$。

%-------------------------------------------------------------------------------
\section{实验评估}
%-------------------------------------------------------------------------------

\subsection{实验设置}

我们在LLaMA-7B和GPT-2(1.5B)模型上评估。私有数据包括来自TOFU~\cite{tofu2024}
基准测试的1000个金丝雀样本。公共数据是C4的10K子集,用于计算$S_{gen}$。

\subsection{评估指标}

\begin{itemize}
\item \textbf{记忆移除}:金丝雀提取率(目标:$<$ 1\%)
\item \textbf{泛化保持}:MMLU(57个任务)和GSM8K(数学推理)准确率
\item \textbf{几何验证}:线性模式连通性(LMC)、纠缠比$\eta$、Hessian谱密度
\end{itemize}

\subsection{结果}

\begin{table}[h]
\centering
\small
\begin{tabular}{lccc}
\hline
方法 & 金丝雀率 & MMLU & GSM8K \\
\hline
基线(无移除) & 98.5\% & 45.2 & 32.1 \\
朴素剪枝 & 12.3\% & 38.1 & 25.4 \\
LibOrtho v1.0(对角) & 8.7\% & 41.2 & 28.9 \\
\textbf{LibOrtho v2.0(K-FAC+OGD)} & \textbf{0.8\%} & \textbf{44.8} & \textbf{31.5} \\
\hline
\end{tabular}
\caption{LLaMA-7B上的记忆移除有效性和能力保持。}
\label{tab:results}
\end{table}

LibOrtho v2.0实现了95\%+的记忆移除(金丝雀率从98.5\%降至0.8\%),同时保
持了98\%+的泛化(MMLU: 44.8 vs. 45.2,GSM8K: 31.5 vs. 32.1)。K-FAC方法
显著优于对角Hessian近似。

\subsection{几何分析}

我们使用多个指标验证几何分离:

\textbf{线性模式连通性(LMC)}:遵循~\cite{lmc2020},我们在$W_{base}$和
$W_{clean} = W_{base} + \Delta W_{gen}$之间沿路径$W(t) = (1-t) W_{base} +
t W_{clean}$插值。损失障碍$\max_t L(W(t)) - \min(L(W_{base}), L(W_{clean}))$
$< 0.01$,确认两个模型位于同一损失盆地中。

\textbf{纠缠比}:我们计算$\eta = \frac{|\langle \Delta W_{gen}, \Delta
W_{mem} \rangle|}{|\Delta W_{gen}||\Delta W_{mem}|}$。对于基线微调,
$\eta = 0.23$(显著纠缠)。使用OGD,$\eta = 0.02$(接近完美正交)。

\textbf{Hessian谱密度}:我们分别计算$\Delta W_{gen}$和$\Delta W_{mem}$的
Hessian特征值。$\Delta W_{mem}$表现出重尾分布,具有离群值($\lambda >
10^3$),而$\Delta W_{gen}$显示Marchenko-Pastur主体分布($\lambda <
10^2$),验证了基于曲率的分离。

\textbf{TOFU金丝雀提取}:遵循~\cite{tofu2024},我们测量模型重现特定金丝
雀短语的能力。基线达到98.5\%的提取率。移除后,这降至0.8\%,证明了有效的
记忆擦除。

%-------------------------------------------------------------------------------
\section{讨论与局限性}
%-------------------------------------------------------------------------------

\subsection{理论洞察}

我们的结果支持泛化与记忆可以分离的几何观点,但需要适当考虑:(1) 参数结
构(K-FAC),(2) 动态约束(OGD),和(3) 基于秩的语义(SVD)。对角近似的
失败突出了捕捉参数交互的重要性。

\subsection{局限性}

我们的方法假设$S_{gen}$可以从公共数据中识别。在实践中,如果私有和公共分
布显著不同,投影可能不是最优的。此外,K-FAC假设Kronecker结构,这可能不
适用于所有层类型。

基于SVD的分离依赖于能力的低秩假设。对于需要高秩更新的任务(例如,学习
许多不同事实),我们的方法可能效果较差。

\subsection{未来方向}

未来的工作应该探索:(1) 损失景观的黎曼几何(超越线性子空间),(2) 用于
语义级分离的因果追踪集成,和(3) 基于任务复杂度的自适应$k$选择。

%-------------------------------------------------------------------------------
\section{结论}
%-------------------------------------------------------------------------------

我们提出了LibOrtho v2.0,一个用于在LLM中分离泛化与记忆的几何原理框架。
通过结合K-FAC结构化曲率、动态OGD和任务向量算术,我们实现了接近完美的
记忆移除,且能力损失最小。我们的工作表明,机制性纠缠可以通过适当的架构
和算法设计来克服,为隐私保护的LLM部署开辟了道路。

%-------------------------------------------------------------------------------
\section*{致谢}
%-------------------------------------------------------------------------------

\textbf{请勿在提交中包含任何可能使您身份暴露的致谢(例如,由于您承认的特
定机构或资助)}

%-------------------------------------------------------------------------------
\cleardoublepage
\appendix
\section*{伦理考量}
%-------------------------------------------------------------------------------

这项工作通过实现选择性移除记忆的私有数据来解决LLM部署中的隐私问题。然
而,我们的方法可能被误用于移除安全护栏或审查特定信息。我们强调,记忆移
除应仅应用于用户提供的私有数据,而不是安全训练或公共知识。

我们承认,由于信息论限制,完美的记忆移除在理论上是不可能的,我们95\%+
的有效性可能仍会留下残留痕迹。用户不应仅依赖技术解决方案,还应采用法律
和政策框架来保护隐私。

所有实验均在公开可用的模型和合成金丝雀数据上进行。我们的评估中未使用真
实的私有用户数据。

%-------------------------------------------------------------------------------
\cleardoublepage
\section*{开放科学}
%-------------------------------------------------------------------------------

为了促进可重现性,我们提供以下工件:

\begin{itemize}
\item \textbf{代码}:LibOrtho v2.0的实现,包括K-FAC计算、OGD训练循环和
  基于SVD的分离。可在以下位置获取:\texttt{[匿名仓库URL]}

\item \textbf{模型}:LLaMA-7B和GPT-2的预计算$S_{gen}$子空间,以及清理后
  的模型检查点。可在以下位置获取:\texttt{[匿名模型中心URL]}

\item \textbf{数据集}:TOFU金丝雀样本和评估脚本。可在以下位置获取:
  \texttt{[匿名数据集URL]}

\item \textbf{实验}:完整的实验配置文件和分析笔记本。可在以下位置获取:
  \texttt{[匿名仓库URL]}

\end{itemize}

所有工件将在接受后公开提供。审稿人可以通过提交系统中提供的匿名链接访问
它们。

%-------------------------------------------------------------------------------
\cleardoublepage
\bibliographystyle{plain}
\bibliography{\jobname}

%%%%%%%%%%%%%%%%%%%%%%%%%%%%%%%%%%%%%%%%%%%%%%%%%%%%%%%%%%%%%%%%%%%%%%%%%%%%%%%%
\end{document}
%%%%%%%%%%%%%%%%%%%%%%%%%%%%%%%%%%%%%%%%%%%%%%%%%%%%%%%%%%%%%%%%%%%%%%%%%%%%%%%%

